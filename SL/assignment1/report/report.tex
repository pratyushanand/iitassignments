\documentclass[a4paper,10pt]{report} %for type of document
\usepackage[utf8x]{inputenc} %for input technique
\usepackage[pdftex]{graphicx} %for inserting images

% Title Page
\title{Software Lab \\Assignment-1}
\author{Pratyush Anand \\2010EEY7515 \\C.Tech, IITD}


\begin{document}
  \maketitle

  \chapter{Problem Statement}
    Write code for various matrix operations.

  \chapter{Specifications}
    \begin{enumerate}
      \item Select input from user for matrix operation to be performed.
      Like mult (for multiplication) add (for addition) etc.
      \newline
      e.g. ./matrix -o mult
      \item Select size of matrix from user as row and column for first
      and second matrix.
      \newline
      e.g. ./matrix -o mult -i 3 -j 5 -k 5 -l 2
      \item Now enter numbers separated by comma(,). 
               Here code will make some assumption. Like for
	       multiplication, if size for first matrix is 3x5 and for
	       second matrix is 5x2 then number of elements
	       entered must be 25. If numbers are less than 25 then rest
	       of the number will be assigned to the value 0. First 5 numbers
	       would be elements of first row of first matrix, next 5 for
	       the second row of first matrix and next 5 for the last
	       row. Division for the second matrix will be done in the chunk of 2.
	       i.e. after the 15th number, first 2 number will be assigned for row 1
	       and so on.
      \newline
	So a typical command line operation will be like this:
      \newline
	./matrix -o mult -i 3 -j 5 -k 5 -l 2 -v 1,2,3,4,6,7,8,9,10,11,12,13,14,15,16,17,18,
	19,20,21,22,23,24,25
    \end{enumerate}

  \chapter{Operation}
  Required memory will be allocated dynamically as per the size received from user,
  Allocated memory will be memset to zero or can be allocated by calloc.
      \newline
  Two arrays will be created as first and second matrix. Values received
  from the user will be stored in these arrays.
      \newline
  Finally a separate function will be created for each supported
  operation and the same will be called from main routine. Pointer of
  first element of both matrix , pointer of output matrix and row and column
  size will be passed to the operation subroutine in case of addition and subtraction.
  Column size of second matrix will also be needed for multiplication
  operation.
      \newline
  After returning from this subroutine main function will print
  all the output at console.
      \newline
  Finally it will free all the dynamically allocated memory.

    \section{Software Tools Required}
    The following software tools will be used:
    \begin{itemize}
    \item pdflatex
    \item vim editor
    \item gcc
    \item ddd
    \end{itemize}
    pdflatex will be used to compile the tex file and vim will be used to edit
    the tex file.
    gcc will be used to compile c code and ddd will be used to debug.
    
\end{document}          
