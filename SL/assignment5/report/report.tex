\documentclass[a4paper,10pt]{report} %for type of document
\usepackage[utf8x]{inputenc} %for input technique
\usepackage[pdftex]{graphicx} %for inserting images

% Title Page
\title{Software Lab \\Assignment-5}
\author{Pratyush Anand \\2010EEY7515 \\C.Tech, IITD}


\begin{document}
  \maketitle

  \chapter{Problem Statement}
    Allocation of vacant sheets available in different departments
\chapter{Assumptions}
\begin{itemize}
  \item There would be an input file with name of different departments and total available sheets in the department.
  \item Dean can add CGPA for a student, but he can not modify the data once entered.
  \item All the students will  be allowed to enter their choices before round1 allocation happens.
  \item There would be no login password for simplicity of program.
  \item Admin will enter correct name and marks. will not be rectified
  \item Student will npt be allowed to modify his choices.
  \item Admina and student will only enter one of the displayed menu.
  once entered.
\end{itemize}
\chapter{Specifications}

\begin{itemize}
 \item Dean can login with admin.
 \item Dean can enter CGPA against all student ID.
 \item He can update initial vacancy.
 \item He can ask for allocation by specifying round number.
 \item He can publish result.
 \item Student can login with his ID.
 \item He can enter his 10 choices along with one extraordinary choice.
\item Dean can lock the system for entering choice detail (may be after last date)

\end{itemize}


\chapter{Operations}
\begin{itemize}

  \item First of all an empty cvs repository will be created.
  \item After developing basic framework, code will be committed. There after each successive addition will be committed at stages.
  \item Admin can load CGPA table from a file. He can add an entry to the table. Added entry will be inserted into the table as per
descending order of CGPA. He can finally save table to the file. In such process, we always have a list sorted in descending order of CGPA.
\item A simple linked list will be created to implement it. Link will be traversed lineraly and a node will be added at the appropriate position.
\item In each node we have following elements.
\begin{enumerate}
 \item student id
 \item his cgpa
 \item array of his 11 choices
\end{enumerate}
\item admin can also update another list with department code and vacancy.
\item Now when student enters his login id, list will be traversed as per his login ID and will be updated as per choices entered.
\item Now for each round software will do the allocation as per defined criteria. It will read first entry from the student table and then it will look
into vacancy table as per his choices. If there is a vacancy matching to his choice, then it will allocate it for him and will update vacancy table.
\item If allocation happens then he will be marked as allocated.Loop will again start from the first non-allocated node of the list and it will try to
 allocated for all not marked as allocated.
\item In case of no allocation, it will just switch to next node.
\item At the end of allocation for all nodes , it will do one more round for extraordinary preference allocation.
 \end{itemize}

\end{document}          
