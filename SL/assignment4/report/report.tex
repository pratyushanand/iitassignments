\documentclass[a4paper,10pt]{report} %for type of document
\usepackage[utf8x]{inputenc} %for input technique
\usepackage[pdftex]{graphicx} %for inserting images

% Title Page
\title{Software Lab \\Assignment-4}
\author{Pratyush Anand \\2010EEY7515 \\C.Tech, IITD}


\begin{document}
  \maketitle

  \chapter{Problem Statement}
    Implementation of Mathematical card spell out trick

\chapter{Assumptions}
  User will be prompted to select one of the input card.

\chapter{Specifications}
\begin{enumerate}
 \item Input of cards will be read from provided input.txt.
 \item Name of all the input card will be printed on screen.
 \item Now user will be promted to select one of the card.
 \item Three consecutive output will be placed in ouput1.txt, output2.txt, ouput3.txt in the same format as input.txt.
 \item Code Will have comments as per doxygen standard.
 \item Code will be released with cvs repository having information of all incremental development stages.
\end{enumerate}

\chapter{Operations}
\section{General}
\begin{description}
  \item First of all an empty cvs repository will be created.
  \item After developing basic framework, code will be commited. There after each succesive addition will be commited at stages.
 \end{description}
\section{Data Structure and Algorithm}
\begin{enumerate}
 \item We will use two data structures, stack and queue.
\item  Each element from the input list will be read and inserted into queue.
\item Now number of character in the first user input will be read. Lets say this number is n.
\item n number of element from queue will be removed and will be pushed to a temorary stack.
\item Now all the n elements from stack will be poped from stack and will be insrted into queue.
\item Above three steps will be repeated for each input word and all elements of queue will be saved in outputfile.
\end{enumerate}


\end{document}          
