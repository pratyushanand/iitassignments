% Chapter 3

\chapter{Embedded Implementation} % Write in your own chapter title
\section {Evaluation Parameters}
\label{Chapter3}
We have implemented our algorthm on an embedded system (single board
computer)and evaluated it on the following points:
\begin{itemize}
 \item \textbf{Low cost:} It is very important for any mass
 market product.
 \item \textbf{Low power:} Reducing energy consumption is a
 movement and is necessary for a greener world.
 Technically also it is very important for a prolonged
 battery life.
 \item \textbf{Computational power:} Selected platform must
 provide sufficient computational power to perform
 needed software operations.
 \item \textbf{Rapid software development:} Today lot of software
 is available in every domain of technology by open source
 community. A suitable hardware platform must be developed by considering
 re-usability of available software in public domain.
\end{itemize}
\section {Hardware Evaluation}
\indent Most of the commercial embedded systems use the following
popular CPUs:
\begin{itemize}
 \item \textbf{MIPS:} The microprocessor without interlocked
 pipeline stages(MIPS) is based on reduced instruction
 set computer(RISC) architecture. It has been used in
 embedded products mainly for gaming and networking. Many
 operating system like Windows CE, QNX and Linux etc
 supports this architecture.
 \item \textbf{AVR32:} AVR is also a 32 bit RISC architecture CPU
 designed by Atmel. It is suited for low power and high
 code density applications. It is also supported by
 Linux. Further, it supports hardware accelerator for
 JAVA byte code.
 \item \textbf{PPC:} Power performance PC, known as PowerPC is
 also a RISC architecture CPU. It is very popular in
 automotive, defence, networking market. However, it is not
 suitable for low power applications. It is also
 supported by Linux.
 \item \textbf{Atom:} Atom is an ultra low voltage x86
 architecture from Intel. It is also very power
 efficient. It has been used in many notebook and mobile
 phone applications. Further, it is well supported by many
 operating system including Linux.
 \item \textbf{ARM:} Advance RISC machine (ARM) is again a RISC
 architecture based CPU. As of now, it is the most
 popular CPU having embedded CPU market share of more
 than two third. It is available in both 32 bit and 64
 bit versions. It has been used in product ranging from
 notebook, mobile phones, gaming devices, networking
 devices, low power storage server to almost all the
 domain of embedded applications. It is also supported by
 various operating systems including Linux.
\end{itemize}
We decided to use ARM based SOC by looking its advantages of low power
and strong support base. There are various evaluation boards available
in the market by different vendors having ARM CPUs. Few examples, BCM
family from Broadcom, SPEAr13xx family from STMicroelectronics, OMAP,
AM335x, Davinci families from Texas instruments, Tegra family from
Nvidia, Exynos family from Samsung, Kirkwood, Orion5x families from
Marvel etc. Out of all these, BeagleBoard and BeagleBone based on TI
OMAP and AM335x SOCs respectively and Raspberry Pi based on Broadcom
BCM2835 SOC are very popular among educational hobbyist. Both of these
boards are available with ARM Ubuntu support. Raspberry Pi has ARM11
core which is ARMv6 and Beagle has Cortex A8 which is ARMv7. Since we
wanted to evaluate performance of selected algorithms at a low end CPU,
therefore we carried our experiments with Raspberry Pi.
\section {Software Evaluation}
\indent Developing a complete code from scratch and without using any OS
or with a very light weight kernel will have advantages in terms of boot
and execution time. However, code development would be very slow. Further,
all evaluated applications including our own developed use
OpenCV library, therefore we looked for the operating systems which
supports OpenCV. There are only two options for embedded platforms,
Windows CE and Linux. Both windows CE and Linux supports ARM
architecture, but windows CE is not free while Linux is open source.
Development over Linux has several other advantages:
\begin{itemize}
 \item Applications are portable from one architecture to
 another. Which means an applications developed on x86
 platform must also work on ARM platform.
 \item Lot of other people are working with it, so a strong
 support.
 \item Code can easily be reused by next developer.
 \item Code debugging is easier.
\end{itemize}
\section {Components of embedded Linux software}
In this section we describe about the necessary tools and software
components that we need to use / develop for complete application.
\subsection {Compiler}
\indent We use native GNU C Compiler (GCC) when code has to be executed
on x86 platform. However, we can not use the same GCC compiler at x86 if
code has to be executed on another architecture say ARM. In such case we
need a cross compiler. This cross compiler is an executable binary which
executes itself at native platform (x86) but, produces another
binary which is executable at target platform (ARM).\\
\indent Preparing a cross compiler from scratch is a tedious job. One
will need source of Linux kernel, binutils, glibc and GCC. Further,
these different sources are released independently and finding a
compatible version is another challenge. However, in most of the cases
we do not need to go through these steps. A ready to use cross toolchain
for ARM and other target is already available in the public domain.\\
\indent If we already have target ready with Linux bootable binaries,
then we can even avoid using cross compiler at all. Now a native GCC is
also available for ARM which can be used to compile Linux Kernel as well
as applications. However, compilation time of Linux kernel with the
native compiler is quite huge and therefore, its use is not preferred.
\subsection {BootROM}
\indent An ARM architecture system boots either from 0x00000000 or from
0xFFFF0000 depending on default vector mode of SOC as low vectored or
high vectored respectively. It is necessary to have a valid code at
default boot location. Normally, most of the SOCs provide hard fixed
code lying in ROM at reset boot location. This piece of software is
known as BootROM. However, there are SOCs which allows to connect
external ROM/Flash memories at reset boot address. \\
\indent This minimal software allows the system to load bigger binaries
from different sources. SOCs are available with boot from USB device, SD
card, NOR / NAND Flash, PCIe device etc. A particular boot mode is
selected using combination of user GPIO pins. In all the boot mode,
common practice is used as follows. BootROM gets first and second level
boot code from external host / device and executes it in SOC's internal
SRAM and external DRAM respectively. Depending on the boot mode ,
bootROM code behaves either as master or slave.\\
\indent For example, bootROM works as slave in USB device mode. USB
device port of SOC is connected to the USB host port of PC. A custom
driver at PC enumerates this bootROM device. BootROM USB device normally
waits for an executable binary from host PC in a loop. New binary is
executed as soon as it is received.\\
\indent BootROM works as master in SD boot mode. It initializes SD
controller and then looks for predefined executables in SD device. If
a valid executable is found, then it is executed.
\subsection {First level boot loader}
\indent RAM is an important part of embedded system which is needed as
run time memory. Normally SOC is designed with some amount of internal
SRAM. SRAM memory is directly accessible and no software initialization
is needed. However, it is very costly. Therefore, connecting even
external SRAM for extra needed RAM is not advised. When an embedded
software executable needs bigger amount of main memory such as in Mega
Bytes, then use of DRAM is preferred. However, DRAM is not directly
accessible by the CPU. First level boot loader has a role to
initialize DRAM controller, so that memory is transparently visible to
CPU. In some cases, first level boot loader itself loads second level
boot loader, while in some other control is passed back to bootROM which
loads second level boot loader. There can be few light weight system
where, second level boot loader is avoided and kernel is directly
loaded.
\subsection {Second level boot loader}
\indent Second level boot loader has following responsibilities:
\begin{itemize}
 \item Provides a debug console till kernel initializes
 console driver. Normally UART is used as initial debug
 console in an embedded environment.
 \item Initializes few more hardware needed for kernel booting.
 \item Provides boot arguments to the kernel. These arguments
 allow kernel and file system to load in a particular
 specification.
 \item Brings kernel to the main memory.
 \item Transfers control to kernel for further execution.
\end{itemize}
\subsection {Kernel}
\indent Kernel is the core of software system. It has following primary
responsibilities:
\begin{itemize}
 \item It initializes all the needed hardware resources like
 CPU, camera, USB, DMA, memory, network controller,
 console and other various input and output devices.
 Linux has huge advantage here. It supports most of the
 hardware available in the market.
 \item It manages cache sync when main memory is used by more
 than one master such as DMA and CPU.
 \item Kernel controls the execution of different task. It
 schedules them on the basis of priorities of the task
 and chosen scheduling algorithm.
 \item Kernel program as well as user space application need huge
 amount of dynamic memory. These memories are allocated
 and freed at run time. It is also very important that
 they are well protected and any illegal access is
 appropriately reported. Kernel takes care of all these
 by proper memory management, allocation and
 deallocation.
 \item It supports various user space file system.
 \item Finally it calls first user space function called init.
\end{itemize}
\subsection {Root filesystem}
\indent File system defines the way a file can be stored in the memory.
Root file system is the primary file system where root directory is
located. All other file system can be mounted within a sub-directly of
this root file system. Linux kernel supports file system like etx2,
ext3, ext4, ntfs , fat32 etc. However, embedded Linxu uses cramfs,
jffs2, yaffs2 which are more suitable for flash. Network File System
(NFS) is also supported by Linux which is very useful during kernel
debug phase.
\section {Application execution at Raspberry Pi}
BootROM of Raspberry Pi provides boot from SD card. Its boot process can
be summarized as under:
\begin{itemize}
 \item It initializes SD controller and looks for any connected
 SD device.
 \item If a device is found, it looks for a valid FAT32 file
 system.
 \item Then it searches for $bootcode.bin$, which is the first
 level boot loader. This binary file is copied to L2
 cache. L2 cache is treated here as SRAM. Execution of
 this binary does the job of SDRAM initialization.
 \item Once SDRAM is available, $start.elf$ is copied from SD
 disk to SDRAM and is executed from there. $start.elf$
 functions as second level boot code. It reads
 $cmdline.txt$ and $config.txt$ and passes boot arguments to
 kernel accordingly.
 \item Second level boot loader copies $kernel.img$ at 0x8000.
 \item Till this point everything was executed at GPU of BCM2837.
 \item $kernel.img$, is the first binary that runs on the ARM
 processor.
 \item When kernel is booted, it looks for ext4 root file system at
 the second partition of SD card. If a valid root file
 system is found, it executes $/sbin/init$ , the first user
 process.
\end{itemize}
\subsection {Raspberry Pi kernel compilation}
\begin{itemize}
 \item RPI kernel git repository is cloned.\par
 \framebox[1.1\width]{git clone https://github.com/raspberrypi/linux.git} \par
 \item Working tag / branch checked out.\par
 \framebox[1.1\width]{git checkout -b rpi\_working origin/rpi-3.6.y} \par
 \item Following default configurations are available in
 $arch/arm/configs$ directory for RPI platform. 
 \begin{itemize}
 \item bcmrpi\_cutdown\_defconfig
 \item bcmrpi\_defconfig
 \item bcmrpi\_emergency\_defconfig
 \item bcmrpi\_quick\_defconfig
 \end{itemize}
 Cut down configuration is selected for a lighter
 kernel.\par
 \framebox[1.1\width]{make ARCH=arm CROSS\_COMPILE=arm-linux-gnueabi- bcmrpi\_cutdown\_defconfig} \par
 $arm-linux-gnueabi-$ is ARM cross compiler provided by Ubuntu Linux package. 
 \item Above $make$ prepares a $./config$ for kernel configuration.
 This default configuration can be modified further by
 $menuconfig$ operation, if needed.\par
 \framebox[1.1\width]{make ARCH=arm CROSS\_COMPILE=arm-linux-gnueabi- menuconfig } \par
 \item Then the kernel image is prepared.\par
 \framebox[1.1\width]{make ARCH=arm CROSS\_COMPILE=arm-linux-gnueabi- } \par
 Two kernel images are prepared in $arch/arm/boot$ directory by
 above command. Kernel image with name $Image$ is
 uncompressed and with name $zImage$ is compressed. RPI
 boot loader $start.elf$ expects a uncompressed kernel
 image. Therefore copy $arch/arm/boot/Image$ to SD card.
 If host PC has multiple CPU then -j option can be used
 for faster compilation. Following command will use four
 CPUs in parallel for compilation.\par
 \framebox[1.1\width]{make ARCH=arm CROSS\_COMPILE=arm-linux-gnueabi- j4 } \par
 \item If a driver is selected to be loaded as module at run time
 then, modules can be prepared as follows:\par
 \framebox[1.1\width]{make ARCH=arm CROSS\_COMPILE=arm-linux-gnueabi- modules} \par
 Install all modules to $/lib/modlues/$ directory of ext4
 partition of SD card.\par
 \fbox{\parbox{\textwidth}{%
 make ARCH=arm CROSS\_COMPILE=arm-linux-gnueabi- \\
 \hspace*{40 mm}INSTALL\_MOD\_PATH=/tmp/ modules\_install\\
 cp -r /tmp/lib/modules/* /mnt/lib/modules/
 }} \par
 Here, ext4 partition of SD card has been mounted to $/mnt$.
\end{itemize}
