% Chapter 3

\chapter{Embedded Implementation} % Write in your own chapter title
\label{Chapter3}
We have done evaluation of embedded solution of our implementation on
the basis of following:
\begin{itemize}
	\item \textbf{Low cost:} It is very important for any mass
		market product.
	\item \textbf{Low power:} Reducing energy consumption is a
		movement and is necessary for a greener world.
		Technically also it is very important for a prolonged
		battery life.
	\item \textbf{Computational power:} Selected platform must
		provide sufficient computational power to perform
		needed software operations.
	\item \textbf{Rapid software development:} Today lot of software
		is available in every domain of science by open source
		community. Platform must be developed by considering
		re-usability of available software in public domain.
\end{itemize}
\section {Hardware Evaluation}
\indent Most of the commercial embedded systems uses following CPUs.
\begin{itemize}
	\item \textbf{MIPS:} The microprocessor without interlocked
		pipeline stages(MIPS) is based on reduced instruction
		set computer(RISC) architecture. It has been used in
		embedded products mainly for gaming and networking. Many
		operating system like Windows CE, QNX and Linux etc
		supports this architecture.
	\item \textbf{AVR32:} AVR is also a 32 bit RISC architecture CPU
		designed by Atmel. It is suited for low power and high
		code density applications.  It is also supported by
		Linux. Further, it supports hardware accelerator for
		JAVA byte code.
	\item \textbf{PPC:} Power performance PC, known as PowerPC is
		also a RISC architecture CPU. It is very popular in
		automotive, defence, networking market. However, it not
		suitable for low power applications. It is also
		supported by Linux.
	\item \textbf{Atom:} Atom is a ultra low voltage x86
		architecture from Intel. It is also very power
		efficient. It has been used in many notebook and mobile
		phone applications. Further, well supported by many
		operating system including Linux.
	\item \textbf{ARM:} Advance RISC machine (ARM) is again a RISC
		architecture based CPU. As of now, it is the most
		popular CPU having embedded CPU market share of more
		than two third. It is available in both 32 bit and 64
		bit versions. It has been used in product ranging from
		notebook, mobile phones, gaming devices, networking
		devices, low power storage server to almost all the
		domain of embedded applications. It is also supported by
		various operating systems including Linux.
\end{itemize}

We decided to use ARM based SOC by looking its advantages of low power
and strong support base. There are various evaluation boards available
in the market by different vendors having ARM CPUs. Few examples, BCM
family from Broadcom, SPEAr13xx family from STMicroelectronics, OMAP,
AM335x, Davinci families from Texas instruments, Tegra family from
Nvidia, Exynos family from samsung, Kirkwood, Orion5x families from
Marvel etc. Out of all these, BeagleBoard and BeagleBone based on TI
OMAP and AM335x SOCs respectively and Raspberry pi based on Broadcom
BCM2835 SOC are very popular among educational hobbyist. Both of these
boards are available with ARM ubuntu support. Raspberry pi has ARM11
core which is ARMv6 and Beagle has Cortex A8 which is ARMv7. Since we
wanted to evaluate performance of selected algorithms at a low end CPU,
therefore we carried our experiments with Raspberry pi.

\section {Software Evaluation}
\indent Developing a complete code from scratch and without using any OS
or with a very light weight kernel will have advantage in terms of boot
and execution time. However, development will be very slow. Further,
all evaluated applications including our own developed use
OpenCV library, therefore we looked for the operating systems which
supports OpenCV. There are only two options for embedded platforms,
Windows CE and Linux. Both windows CE and Linux supports ARM
architecture, but windows CE is not free while Linux is open source.
Development over Linux has several other advantages:
\begin{itemize}
	\item Applications are portable from one architecture to
		another. Which means an applications developed on x86
		platform must also work on ARM platform.
\end{itemize}
\section {Component of embedded Linux software}
\subsection {Toolchain}
cross toolchain, why and how
\subsection {BootROM}
why and how, if not then other alternatives.
\subsection {First level boot loader}
ddr initialization code, why and how
\subsection {Second level boot loader}
preparation of OS booting, why and how
\subsection {Second level boot }
preparation of OS booting, why and how
\subsection {Operation System}
preparation of Linux booting. Minimal Linux component needed
(menuconfig) for our implementation. Also discuss about V4L2 camera
framework.
\subsection {Root filesystem}
RFS, what, why and how
